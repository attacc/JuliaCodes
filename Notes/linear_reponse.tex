%\documentclass[twocolumn,showpacs,prb,superscriptaddress,aps,floatfix]{revtex4-1}
\documentclass[preprint,showpacs,prb,superscriptaddress,aps,floatfix]{revtex4-1}
\usepackage{rotating}
\usepackage{amsmath}
\usepackage{color}
\usepackage{graphicx}
\usepackage{epsfig}
\usepackage{courier}
\usepackage[active]{srcltx}
\usepackage[sort&compress]{natbib}


\newcommand{\uu}{{\bf u}}
\newcommand{\qq}{{\bf q}}
\newcommand{\ab}{{\bf a}}
\newcommand{\bb}{{\bf b}}
\newcommand{\hh}{{\bf h}}
\newcommand{\rr}{{\bf r}}
\newcommand{\pp}{{\bf p}}
\newcommand{\PP}{{\bf P}}
\newcommand{\RR}{{\bf R}}
\newcommand{\kk}{{\bf k}}
\newcommand{\HH}{{\bf H}}
\newcommand{\GG}{{\bf G}}
\newcommand{\SiS}{{\bf \Sigma}}
\newcommand{\VV}{{\bf V}}
\newcommand{\UU}{{\bf U}}
\newcommand{\w}{\omega}
\newcommand{\tf}{\textbf}
\newcommand{\bo}{\mathbf}
\newcommand{\br}{{\bf r}}
\newcommand{\be}{\begin{equation}}
\newcommand{\ee}{\end{equation}}
\newcommand{\ben}{\begin{equation*}}
\newcommand{\een}{\end{equation*}}
\newcommand{\bea}{\begin{eqnarray}}
\newcommand{\eea}{\end{eqnarray}}
\newcommand{\bean}{\begin{eqnarray*}}
\newcommand{\eean}{\end{eqnarray*}}
\newcommand{\nup}{n_{\uparrow}}
\newcommand{\ndown}{n_{\downarrow}}
\newcommand{\Id}[1] {\int \! \! {\rm d}^3 #1}
\renewcommand{\v}[1]{{\bf #1}}
\renewcommand{\[}{\left[}
\renewcommand{\]}{\right]}
\renewcommand{\(}{\left(}
\renewcommand{\)}{\right)}
\def\efield{\boldsymbol{\cal E}} 
\def\ket#1{\vert#1\rangle}
\def\bra#1{\langle#1\vert}
\def\pw{^{({\rm W})}}
\def\ph{^{({\rm H})}}
\def\D{{D}\ph}




 
\newcommand{\grenoble}{Institut N\'eel,
CNRS/UJF, 25 rue des Martyrs BP 166, B\^{a}timent D 38042 Grenoble
cedex 9 France} 
\newcommand{\rome}{Istituto di Struttura della Materia (ISM), Consiglio Nazionale delle Ricerche, Via Salaria Km 29.5, CP 10, 00016 Monterotondo Stazione, Italy}
\newcommand{\coimbra}{Centre for Computational Physics and Physics Department, University of Coimbra, Rua Larga, 3004-516 Coimbra, Portugal}

\begin{document}
\title{Linear response from EOM of the density matrix}
\author{C. Attaccalite}
\affiliation{\grenoble}

\begin{abstract}
In these notes I sketch the linear response from density matrix equation of motion in length gauge
\end{abstract}           

\maketitle
When an external perturbation $U(t)$ is switched on it induces a variation of the density matrix, 
$\Delta \rho(\kk,t)$.
In the case of a strong applied laser field these changes depend on
all possible orders in the external field. However for weak fields the
linear term is dominant. 
Proceeding similarly to Ref.~\cite{bsedynamic} we consider the retarded density-density correlation function:
\be
\label{chi-rr}
\chi^{\mathrm{r}}(\rr,t;\rr',t') =
-i\[\langle \rho(\rr,t)\rho(\rr',t')\rangle 
- \langle \rho(\rr,t)\rangle \langle \rho(\rr',t')\rangle\]\theta\(t-t'\).
\ee
$\chi^{\mathrm{r}}$ describes the linear response of the system to a weak
perturbation, represented in Eq.~\eqref{hamiltonian}  by $U$,
\be
\label{eq:phychi}
\chi^{\mathrm{r}}(\rr,t;\rr',t') = \left. \frac{\langle
  \delta\rho(\rr t) \rangle}{\delta U(\rr^\prime t^{\prime})}\right\vert_{U=0}.
\ee
We start by expanding  $\chi (\mathrm{r})$ in terms of the Kohn-Sham orbitals:
\be
\label{basisChi}
\chi^{\mathrm r}(\rr,t; \mathbf{r'},t'; \mathbf q) =
  \sum_{\substack{i,j,\kk \\ l,m,\kk^\prime} } \chi^{\mathrm r}_{\substack{i,j,\kk \\ l,m,\kk^\prime} }(t,t^\prime; \mathbf q)
\times \varphi_{i,\kk} (\rr)\varphi^*_{j, \mathbf {k+q}} (\rr) \varphi^*_{l, \mathbf {k^\prime}} (\rr')
\varphi_{m, \mathbf {k^\prime+q}} (\rr'),
\ee
where $\qq$ is the momentum, and we define the matrix elements of $\chi^{\mathrm r}$ as,
\be
\label{eq:chimat}
 \chi^{\mathrm r}_{\substack{ij,\kk \\ lm,\kk^\prime} }(t,t^\prime; \mathbf q) =
\iint \mathbf{d}^3r \mathbf{d}^3r^\prime \varphi^*_{i,\kk}(\rr)
\varphi^*_{m, \kk^\prime+\qq} (\rr^\prime)
\varphi_{j,\kk+\qq} (\rr) \varphi_{l, \kk^\prime} (\rr^\prime).
\ee
Since we are interested only in the optical response, in what follows
we restrict ourselves to the case $\qq =0$ and drop the $\qq$
dependence of $\chi^{\mathrm r}$. 
The position operator in periodic system is defined as:\cite{blount1962solid}
\be
\hat \rr = i \frac{\partial}{\partial \kk} + \hat A
\label{rperiodic}
\ee
where $\hat A$ is the transition dipole moment or a generalization of the Berry connection:\cite{silva2019high,wang2006ab}: 
\be
\boldsymbol{A}\pw_{nm}\left(\boldsymbol{k}\right)=i\int_{V_{C}}d\boldsymbol{r}~u^{*}_{n\boldsymbol{k}}\left(\boldsymbol{r}\right)\frac{\partial}{\partial\boldsymbol{k}}u_{m\boldsymbol{k}}\left(\boldsymbol{r}\right).
\ee
The above defintion of $\hat \rr$ can be obtained from the Blount\cite{blount1962solid} or starting from the definition of polarization given by Resta\cite{resta1994macroscopic} and reduction of the many-body wave-function to a single slater determinant.\cite{souza2004dynamics}
Using the definition Eq.\ref{rperiodic} for the position operator, it is possible to write down the Hamiltonian as:
\begin{align}
\hat{\mathcal{H}}\left(t\right) & =\sum_{\boldsymbol{k}\in FBZ}\sum_{n,m}c_{n\boldsymbol{k}}^{\dagger}H_{nm}\left(\boldsymbol{k}\right)c_{m\boldsymbol{k}}\nonumber \\
 & +\sum_{\boldsymbol{k}\in FBZ}\sum_{n,m}c_{n\boldsymbol{k}}^{\dagger}\left|e\right|\boldsymbol{E}\left(t\right).\left[i\delta_{nm}\frac{\partial}{\partial\boldsymbol{k}}+\boldsymbol{A}_{nm}\left(\boldsymbol{k}\right)\right]c_{m\boldsymbol{k}}
\end{align}
where $c_{n\boldsymbol{k}}^{\dagger}$ ($c_{n\boldsymbol{k}}$) is the fermionic creation (annihilation) operator of a Bloch state. Using the above Hamiltonian and the EOMs for $| \psi_\kk \rangle $ and  $\langle \psi_\kk|$, plus the definition of the density matrix operator, we can write down the EOM for the density matrix as:  
\begin{align}
i\hbar\frac{\partial}{\partial t}\rho_{nm}\left(\boldsymbol{k},t\right) & =\left[H\left(\boldsymbol{k}\right),\rho\left(\boldsymbol{k},t\right)\right]_{nm}\nonumber \\
 & +i\left|e\right|\boldsymbol{E}\left(t\right).\frac{\partial}{\partial\boldsymbol{k}}\rho_{nm}\left(\boldsymbol{k},t\right)\nonumber \\
 & +\left|e\right|\boldsymbol{E}\left(t\right).\left[\boldsymbol{A}\left(\boldsymbol{k}\right),\rho\left(\boldsymbol{k},t\right)\right]_{nm}\label{eq:RDM_eq_motion}.
\end{align}
We can derive the linear response using the expantion of $\rho(t)$ for small external fields as:
\be
\rho_{nm}\left(\boldsymbol{k},t\right) = \rho^0_{nm} + \sum_{ij} \int dt' \frac{\partial \rho_{nm}\left(\boldsymbol{k},t\right)}{\partial U_{ij}(t')} \Delta U_{ij}(t') + O(\Delta U^2) 
\label{rhoexp}
\ee
in our case $U$ corresponds to:
\be
 U_{ij}(t) = \left|e\right|\boldsymbol{E}\left(t\right).\left[i\delta_{nm}\frac{\partial}{\partial\boldsymbol{k}}+\boldsymbol{A}_{nm}\left(\boldsymbol{k}\right)\right]
\ee
and we can define the response function $\chi$ as:
\be
\chi_{ij,nm}(\kk,t,t')= \frac{\partial \rho_{nm}\left(\boldsymbol{k},t\right)}{\partial U_{ij}(t')} 
\ee
In order to get an equation for $\chi_{ij,nm}(\kk,t,t')$ we insert expation Eq.~\ref{rhoexp} in Eq.~\ref{eq:RDM_eq_motion} and retain only term to the first order in the external field. Notice that:
\bea
H_{ij}\left(\boldsymbol{k}\right) &=& \epsilon_{i \kk} \delta_{ij} \\
\rho^0_{nm}(\kk) &=& f_{n \kk} \delta_{nm} \\
\partial_\kk \rho^0_{nm}(\kk) &=& 0 
\eea
where $f_{n\kk}=f(\epsilon_{n \kk})$ is the Fermi distribution. Therefore we get:
\bea
\label{dtGtt}
i \frac{\partial}{\partial t} \chi^{\mathrm r}_{\substack{ij,lm, \kk }}(t,t^\prime)=\frac{\delta}{ \delta U_{lm, \pp}(t^{\prime})} [H({\kk}) + \left|e\right|\boldsymbol{E}\left(t\right) \boldsymbol{A}\left(\boldsymbol{k}\right) , \rho(\kk,t) ]_{\substack{ji}}. 
\eea
Notice that the term with the k-derivative $\partial_k \rho(\kk)$ do not contribute to the linear reponse, because it gives a contribution of the second order in $E$. Doing some math we get:
\bean
\label{diag_contr}
\left. \frac{\delta}{ \delta U_{l,m}(t^{\prime})} [H(\kk)+ \UU_{\kk},\rho(\kk,t)]_{\substack{ji}} \right\vert_{U=0}=
(e_{j\kk} - e_{i\kk})  \chi^{\mathrm r}_{\substack{ji, \kk \\ lm }}(t-t^\prime) + i(f_{i\kk}-f_{j\kk})\delta_{jl}\delta_{im} \delta(t-t'). 
\eean
and by Fourier transform we get:
\bea
 \left[  \omega- \left(\epsilon_{j \kk} -\epsilon_{i \kk}\right) \right] \chi^{\mathrm r}_{\substack{ij,\kk\\lm \mathbf p}}(\omega) = i \(f_{i \kk}-f_{j \kk}\) \left[ \delta_{jl} \delta_{im} \delta_{\kk,\pp} \right.
\eea
Now we want to reconstruct the $\chi(\rr,\rr)$ average on the cell and we insert again the position operator keeping only the linear terms and we get:
\be
\chi(\omega) = \sum_{ij} \frac{|A_{ij}(\kk)|^2}{\omega- \epsilon_{j \kk} -\epsilon_{i \kk}}  \(f_{i \kk}-f_{j \kk}\)  = \sum_{ij} \frac{| \langle \psi_{i \kk} | \partial_k \psi_{j \kk} \rangle|^2}{\omega- \epsilon_{j \kk} -\epsilon_{i \kk}}  \(f_{i \kk}-f_{j \kk}\) 
\label{xhiw}
\ee
Using perturbation theory in $\kk$ we can transform this formula in the standard one.\cite{wang2006ab}
We consider an Hamiltonian:
\be
H (\kk + \Delta \kk_\alpha) = H (\kk)  + \Delta \kk_\alpha \partial_\alpha  H (\kk)
\ee
where $\partial_\alpha = \partial/\partial_{\kk_\alpha}$. We consider the second term as a perturbation and write the perturbation theory:
\be
	\ket{ \psi_{n,\boldsymbol{k}+\Delta \kk}}=\ket{ \psi_{n,\boldsymbol{k}}}+\Delta \kk_\alpha \sum_{m} \frac{\bra{\psi_{m,\boldsymbol{k} }} H_\alpha   \ket{\psi_{n,\boldsymbol{k} }}}{\epsilon_{m\kk}-\epsilon_{n\kk}} \ket{\psi_{m,\boldsymbol{k} }}
\ee
where $ H_\alpha =\partial_\alpha  H$. Then we can define the derivative respect to $\kk$ of an eigenvector as:
\be
\ket{ \partial_\alpha \psi_{n,\boldsymbol{k}}}=   \frac{\ket{ \psi_{n,\boldsymbol{k}+\Delta \kk}}-\ket{ \psi_{n,\boldsymbol{k}}}}{\Delta \kk}= \sum_{m \neq n} \frac{\bra{\psi_{m,\boldsymbol{k} }} H_\alpha   \ket{\psi_{n,\boldsymbol{k} }}}{\epsilon_{m\kk}-\epsilon_{n\kk}} \ket{\psi_{m,\boldsymbol{k} }}
\ee
If you substitute this formula in Eq.~\ref{xhiw} you get the standard definition of dipoles in term of the commutator of the Hamiltonian.\\
\be
\chi(\omega) =  \sum_{ij} \frac{| \langle \psi_{i \kk} | \partial_k H(\kk)| \psi_{j \kk} \rangle|^2}{\left (\epsilon_{j \kk} -\epsilon_{i \kk} \right)^2}\frac{ \(f_{i \kk}-f_{j \kk}\)}{\omega- \epsilon_{j \kk} -\epsilon_{i \kk}} 
\label{xhiw}
\ee
\textbf{Nota bene 1:} The above derivation of linear response is beatuful but is not fully correct because we assume the differentiability of $\psi_k$ and $H(\kk)$ respect to $\kk$ and this it is true only in a smooth k-gauge. \cite{silva2019high} \\
\textbf{Nota bene 2:} In the above derivation I used the Bloch function $\psi_\kk$, the same result can be obtained using only the periodic part of the Bloch function $u_\kk$ but in this case the term $\partial_k \rho(\kk)$ cannot be distregarded because it will aquire a contribution linear with the field.\cite{PythTB}
\addcontentsline{toc}{chapter}{Bibliography}
\bibliographystyle{apsrev4-1}
\bibliography{hhg}
\end{document}
