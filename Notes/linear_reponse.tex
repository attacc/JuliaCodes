%\documentclass[twocolumn,showpacs,prb,superscriptaddress,aps,floatfix]{revtex4-1}
\documentclass[preprint,showpacs,prb,superscriptaddress,aps,floatfix]{revtex4-1}
\usepackage{rotating}
\usepackage{amsmath}
\usepackage{color}
\usepackage{graphicx}
\usepackage{epsfig}
\usepackage{courier}
\usepackage[active]{srcltx}
\usepackage[sort&compress]{natbib}


\newcommand{\uu}{{\bf u}}
\newcommand{\qq}{{\bf q}}
\newcommand{\ab}{{\bf a}}
\newcommand{\bb}{{\bf b}}
\newcommand{\hh}{{\bf h}}
\newcommand{\rr}{{\bf r}}
\newcommand{\pp}{{\bf p}}
\newcommand{\PP}{{\bf P}}
\newcommand{\RR}{{\bf R}}
\newcommand{\kk}{{\bf k}}
\newcommand{\HH}{{\bf H}}
\newcommand{\GG}{{\bf G}}
\newcommand{\SiS}{{\bf \Sigma}}
\newcommand{\VV}{{\bf V}}
\newcommand{\UU}{{\bf U}}
\newcommand{\w}{\omega}
\newcommand{\tf}{\textbf}
\newcommand{\bo}{\mathbf}
\newcommand{\br}{{\bf r}}
\newcommand{\be}{\begin{equation}}
\newcommand{\ee}{\end{equation}}
\newcommand{\ben}{\begin{equation*}}
\newcommand{\een}{\end{equation*}}
\newcommand{\bea}{\begin{eqnarray}}
\newcommand{\eea}{\end{eqnarray}}
\newcommand{\bean}{\begin{eqnarray*}}
\newcommand{\eean}{\end{eqnarray*}}
\newcommand{\nup}{n_{\uparrow}}
\newcommand{\ndown}{n_{\downarrow}}
\newcommand{\Id}[1] {\int \! \! {\rm d}^3 #1}
\renewcommand{\v}[1]{{\bf #1}}
\renewcommand{\[}{\left[}
\renewcommand{\]}{\right]}
\renewcommand{\(}{\left(}
\renewcommand{\)}{\right)}
\def\efield{\boldsymbol{\cal E}} 
\def\ket#1{\vert#1\rangle}
\def\bra#1{\langle#1\vert}
\def\pw{^{({\rm W})}}
\def\ph{^{({\rm H})}}
\def\D{{D}\ph}




 
\newcommand{\grenoble}{Institut N\'eel,
CNRS/UJF, 25 rue des Martyrs BP 166, B\^{a}timent D 38042 Grenoble
cedex 9 France} 
\newcommand{\rome}{Istituto di Struttura della Materia (ISM), Consiglio Nazionale delle Ricerche, Via Salaria Km 29.5, CP 10, 00016 Monterotondo Stazione, Italy}
\newcommand{\coimbra}{Centre for Computational Physics and Physics Department, University of Coimbra, Rua Larga, 3004-516 Coimbra, Portugal}

\begin{document}
\title{Linear response from EOM of the density matrix}
\author{C. Attaccalite}
\affiliation{\grenoble}

\begin{abstract}
In these notes I sketch the linear response from density matrix equation of motion in length gauge
\end{abstract}           

\maketitle

The position operator in periodic system is defined as:\cite{blount1962solid}
\be
\hat \rr = i \frac{\partial}{\partial \kk} + \hat A
\label{rperiodic}
\ee
where $\hat A$ is the transition dipole moment or a generalization of the Berry connection:\cite{silva2019high,wang2006ab}: 
\be
\boldsymbol{A}\pw_{nm}\left(\boldsymbol{k}\right)=i\int_{V_{C}}d\boldsymbol{r}~u^{*}_{n\boldsymbol{k}}\left(\boldsymbol{r}\right)\frac{\partial}{\partial\boldsymbol{k}}u_{m\boldsymbol{k}}\left(\boldsymbol{r}\right).
\ee
The above defintion of $\hat \rr$ can be obtained from the Blount\cite{blount1962solid} or starting from the definition of polarization given by Resta\cite{resta1994macroscopic} and reduction of the many-body wave-function to a single slater determinant.\cite{souza2004dynamics}
Using the definition Eq.\ref{rperiodic} for the position operator, it is possible to write down the Hamiltonian as:
\begin{align}
\hat{\mathcal{H}}\left(t\right) & =\sum_{\boldsymbol{k}\in FBZ}\sum_{n,m}c_{n\boldsymbol{k}}^{\dagger}H_{nm}\left(\boldsymbol{k}\right)c_{m\boldsymbol{k}}\nonumber \\
 & +\sum_{\boldsymbol{k}\in FBZ}\sum_{n,m}c_{n\boldsymbol{k}}^{\dagger}\left|e\right|\boldsymbol{E}\left(t\right).\left[i\delta_{nm}\frac{\partial}{\partial\boldsymbol{k}}+\boldsymbol{A}_{nm}\left(\boldsymbol{k}\right)\right]c_{m\boldsymbol{k}}
\end{align}
where $c_{n\boldsymbol{k}}^{\dagger}$ ($c_{n\boldsymbol{k}}$) is the fermionic creation (annihilation) operator of a Bloch state. Using the above Hamiltonian and the EOMs for $| \psi_\kk \rangle $ and  $\langle \psi_\kk|$, plus the definition of the density matrix operator, we can write down the EOM for the density matrix as:  
\begin{align}
i\hbar\frac{\partial}{\partial t}\rho_{nm}\left(\boldsymbol{k},t\right) & =\left[H\left(\boldsymbol{k}\right),\rho\left(\boldsymbol{k},t\right)\right]_{nm}\nonumber \\
 & +i\left|e\right|\boldsymbol{E}\left(t\right).\frac{\partial}{\partial\boldsymbol{k}}\rho_{nm}\left(\boldsymbol{k},t\right)\nonumber \\
 & +\left|e\right|\boldsymbol{E}\left(t\right).\left[\boldsymbol{A}\left(\boldsymbol{k}\right),\rho\left(\boldsymbol{k},t\right)\right]_{nm}\label{eq:RDM_eq_motion}.
\end{align}
We can derive the linear response using the expantion of $\rho(t)$ for small external fields as:
\be
\rho_{nm}\left(\boldsymbol{k},t\right) = \rho^0_{nm} + \sum_{ij} \int dt' \frac{\partial \rho_{nm}\left(\boldsymbol{k},t\right)}{\partial U_{ij}(t')} \Delta U_{ij}(t') + O(\Delta U^2) 
\label{rhoexp}
\ee
in our case $U$ corresponds to:
\be
 U_{ij}(t) = \left|e\right|\boldsymbol{E}\left(t\right).\left[i\delta_{nm}\frac{\partial}{\partial\boldsymbol{k}}+\boldsymbol{A}_{nm}\left(\boldsymbol{k}\right)\right]
\ee
and we can define the response function $\chi$ as:
\be
\chi_{ij,nm}(\kk,t,t')= \frac{\partial \rho_{nm}\left(\boldsymbol{k},t\right)}{\partial U_{ij}(t')} 
\ee
In order to get an equation for $\chi_{ij,nm}(\kk,t,t')$ we insert expation Eq.~\ref{rhoexp} in Eq.~\ref{eq:RDM_eq_motion} and retain only term to the first order in the external field. Notice that:
\bea
H_{ij}\left(\boldsymbol{k}\right) &=& \epsilon_{i \kk} \delta_{ij} \\
\rho^0_{nm}(\kk) &=& f_{n \kk} \delta_{nm} \\
\partial_\kk \rho^0_{nm}(\kk) &=& 0 
\eea
where $f_{n\kk}=f(\epsilon_{n \kk})$ is the Fermi distribution. Therefore we get:
\bea
\label{dtGtt}
i \frac{\partial}{\partial t} \chi^{\mathrm r}_{\substack{ij,lm, \kk }}(t,t^\prime)=\frac{\delta}{ \delta U_{lm, \pp}(t^{\prime})} [H({\kk}) + \left|e\right|\boldsymbol{E}\left(t\right) \boldsymbol{A}\left(\boldsymbol{k}\right) , \rho(\kk,t) ]_{\substack{ji}}. 
\eea
Notice that the term with the k-derivative $\partial_k \rho(\kk)$ do not contribute to the linear reponse, because it gives a contribution of the second order in $E$. Doing some math we get:
\bean
\label{diag_contr}
\left. \frac{\delta}{ \delta U_{l,m}(t^{\prime})} [H(\kk)+ \UU_{\kk},\rho(\kk,t)]_{\substack{ji}} \right\vert_{U=0}=
(e_{j\kk} - e_{i\kk})  \chi^{\mathrm r}_{\substack{ji, \kk \\ lm }}(t-t^\prime) + i(f_{i\kk}-f_{j\kk})\delta_{jl}\delta_{im} \delta(t-t'). 
\eean
and by Fourier transform we get:
\bea
 \left[  \omega- \left(\epsilon_{j \kk} -\epsilon_{i \kk}\right) \right] \chi^{\mathrm r}_{\substack{ij,\kk\\lm \mathbf p}}(\omega) = i \(f_{i \kk}-f_{j \kk}\) \left[ \delta_{jl} \delta_{im} \delta_{\kk,\pp} + \right.
\eea

\addcontentsline{toc}{chapter}{Bibliography}
\bibliographystyle{apsrev4-1}
\bibliography{hhg}
\end{document}
