\documentclass[%
% reprint,
superscriptaddress,
%groupedaddress,
%unsortedaddress,
%runinaddress,
%frontmatterverbose, 
preprint,
preprintnumbers,
%nofootinbib,
%nobibnotes,
%bibnotes,
 amsmath,amssymb,
 aps,
%pra,
%prb,
%rmp,
%prstab,
%prstper,
%floatfix,
%twocolumn,
]{revtex4-2}

\usepackage{xcolor}
\usepackage{bbold}
\usepackage{graphicx}% Include figure files
\usepackage{dcolumn}% Align table columns on decimal point
\usepackage{bm}% bold math
\usepackage[colorlinks,allcolors=blue,urlcolor=blue]{hyperref}
\usepackage[normalem]{ulem}
\newcommand\corr[1]{\textcolor{red}{{{#1}}}}


\newcommand{\uu}{{\bf u}}
\newcommand{\qq}{{\bf q}}
\newcommand{\hh}{{\bf h}}
\newcommand{\rr}{{\bf r}}
\newcommand{\pp}{{\bf p}}
\newcommand{\PP}{{\bf P}}
\newcommand{\kk}{{\bf k}}
\newcommand{\HH}{{\bf H}}
\newcommand{\GG}{{\bf G}}
\newcommand{\SiS}{{\bf \Sigma}}
\newcommand{\VV}{{\bf V}}
\newcommand{\UU}{{\bf U}}
\newcommand{\w}{\omega}
\newcommand{\tf}{\textbf}
\newcommand{\bo}{\mathbf}
\newcommand{\br}{{\bf r}}
\newcommand{\be}{\begin{equation}}
\newcommand{\ee}{\end{equation}}
\newcommand{\ben}{\begin{equation*}}
\newcommand{\een}{\end{equation*}}
\newcommand{\bea}{\begin{eqnarray}}
\newcommand{\eea}{\end{eqnarray}}
\newcommand{\bean}{\begin{eqnarray*}}
\newcommand{\eean}{\end{eqnarray*}}
\newcommand{\nup}{n_{\uparrow}}
\newcommand{\ndown}{n_{\downarrow}}

\newcommand{\bdq}{\hat b^{\dagger}_\qq}
\newcommand{\bq}{\hat b_\qq}

\newcommand{\aald}[1]{\hat a^{\dagger}_{\alpha #1}}
\newcommand{\aal}[1]{\hat a_{\alpha #1}}

\newcommand{\abe}[1]{\hat a_{\beta #1 }}
\newcommand{\abed}[1]{\hat a^{\dagger}_{\beta #1}}



\newcommand{\Id}[1] {\int \! \! {\rm d}^3 #1}
\renewcommand{\v}[1]{{\bf #1}}
\renewcommand{\[}{\left[}
\renewcommand{\]}{\right]}
\renewcommand{\(}{\left(}
\renewcommand{\)}{\right)}
\def\efield{\boldsymbol{\cal E}} 
\def\ket#1{\vert#1\rangle}
\def\bra#1{\langle#1\vert}



\begin{document}

\preprint{APS/123-QED}

\newcommand{\cinam}{CNRS/Aix-Marseille Universit\'e, Centre Interdisciplinaire de Nanoscience de Marseille UMR 7325 Campus de Luminy, 13288 Marseille cedex 9, France}

\title{Floquet in a tight-binding model}
\author{Claudio Attaccalite}
\affiliation{\cinam}

%\date{\today}

\begin{abstract}

We sketch the derivation of Floquet following the work of Ikeda et al. \cite{ikeda2018floquet}
\end{abstract}

\maketitle


%%%%%%%%%%%%%%%%%%%%%%%%%%%%%%%%%%%%%%%%%%%%%%%%%%%%%%%%%%%%%%%%%%%%%%%%%%%%
\section{Introduction}
We start from Eq.~12 of Ref.~\cite{ikeda2018floquet}:
\begin{equation}
\bold H_{mn}^F(k)=(n\Omega \mathbb{I} + Q\sigma_z) \delta_{mn} + i^{m-n} J_{m-n} cos\left [ k - \frac{(m-n)\pi}{2}  \right] \sigma_x
\end{equation}
%%%%%%%%%%%%%%%%%%%%%%%%%%%%%%%%%%%%%%%%%%%%%%%%%%%%%%%%%%%%%%%%%%%%%%%%%%
The corresponding Floquet states are defined as the solution of:
\begin{equation}
    \sum_n \bold H^F_{mn} \chi_n^\alpha(k) = E_\alpha(k) \chi_n^\alpha(k)
\end{equation}
where in our case $\alpha=1,2$. In general the size of the Floquet matrix is $\#modes \cdot \#bands$.
The matrix has the structure described in the Ashwin notes.
From the solution of the Floquet Hamiltonia we can find the eigenmodes of the time-dependent Schoerdinger eq.:
\begin{equation}
	\vec \chi^\alpha(k,t)=e^{E_\alpha(k),t} \sum_n \vec \chi^\alpha_n(k)e^{in\Omega t}
\end{equation}
Notice that in our case $\vec \chi^\alpha(k,t)$ is a vector of dimension 2.
The general expression for a time dependent function in term of Floquet states writes:
\be
\vec \psi(k,t)= \sum_\alpha \omega_\alpha \vec \chi^\alpha(k,t) 
\ee
Now we want to use this basis to describe the EOM when a field $sin(\Omega t)$ is turned on at $t=0$. 
We supposed that only valence orbitals are occupied at $t=0$, namely:
\be
\omega_\alpha= \vec \chi^\alpha_n(k)^\dagger \vec \psi_0(k) \\
\ee
\begin{acknowledgments}
CA. acknowledges ANR project COLIBRI No. ANR-22-CE30-0027 and AMIDEX INDIGENA project.

\end{acknowledgments}
%\bibliographystyle{naturemag}
%\bibliographystyle{mynaturemag}
\bibliography{references}

\end{document}
